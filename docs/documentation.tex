\documentclass[12pt,a4paper]{article}

% Packages
\usepackage[utf8]{inputenc}
\usepackage[T1]{fontenc}
\usepackage{geometry}
\usepackage{graphicx}
\usepackage{hyperref}
\usepackage{listings}
\usepackage{xcolor}
\usepackage{fancyhdr}
\usepackage{titlesec}
\usepackage{tcolorbox}
\usepackage{enumitem}
\usepackage{booktabs}
\usepackage{longtable}
\usepackage{float}

% Page setup
\geometry{
    a4paper,
    left=2.5cm,
    right=2.5cm,
    top=3cm,
    bottom=3cm
}

% Colors
\definecolor{codegreen}{rgb}{0,0.6,0}
\definecolor{codegray}{rgb}{0.5,0.5,0.5}
\definecolor{codepurple}{rgb}{0.58,0,0.82}
\definecolor{backcolour}{rgb}{0.95,0.95,0.92}
\definecolor{primaryblue}{RGB}{0,102,204}

% Code listing style
\lstdefinestyle{mystyle}{
    backgroundcolor=\color{backcolour},   
    commentstyle=\color{codegreen},
    keywordstyle=\color{magenta},
    numberstyle=\tiny\color{codegray},
    stringstyle=\color{codepurple},
    basicstyle=\ttfamily\footnotesize,
    breakatwhitespace=false,         
    breaklines=true,                 
    captionpos=b,                    
    keepspaces=true,                 
    numbers=left,                    
    numbersep=5pt,                  
    showspaces=false,                
    showstringspaces=false,
    showtabs=false,                  
    tabsize=2
}
\lstset{style=mystyle}

% Hyperref setup
\hypersetup{
    colorlinks=true,
    linkcolor=primaryblue,
    filecolor=magenta,      
    urlcolor=cyan,
    pdftitle={Secure Task Management System - Technical Documentation},
    pdfauthor={Hema Sri Puppala},
}

% Header and footer
\pagestyle{fancy}
\fancyhf{}
\fancyhead[L]{\leftmark}
\fancyhead[R]{TurboVets Assessment}
\fancyfoot[C]{\thepage}

% Title formatting
\titleformat{\section}
  {\normalfont\Large\bfseries\color{primaryblue}}{\thesection}{1em}{}
\titleformat{\subsection}
  {\normalfont\large\bfseries}{\thesubsection}{1em}{}

% Document
\begin{document}

% Title Page
\begin{titlepage}
    \centering
    \vspace*{2cm}
    
    {\Huge\bfseries Secure Task Management System\par}
    \vspace{1cm}
    {\Large Full-Stack Assessment Project\par}
    \vspace{2cm}
    
    \begin{tcolorbox}[colback=blue!5!white,colframe=primaryblue,width=0.8\textwidth]
        \centering
        \large
        A modern, role-based task management application\\
        built with NestJS, Angular, and TypeScript
    \end{tcolorbox}
    
    \vspace{3cm}
    
    {\Large\textbf{Author:} Hema Sri Puppala\par}
    \vspace{0.5cm}
    {\large\textbf{Organization:} TurboVets\par}
    \vspace{0.5cm}
    {\large\textbf{Date:} February 2026\par}
    
    \vfill
    
    {\large Technical Documentation\par}
    {\large Version 1.0\par}
\end{titlepage}

% Table of Contents
\tableofcontents
\newpage

% Abstract
\section*{Abstract}
\addcontentsline{toc}{section}{Abstract}

This document provides comprehensive technical documentation for the Secure Task Management System, a full-stack web application developed as an assessment project for TurboVets. The system implements enterprise-grade features including JWT authentication, role-based access control (RBAC), and a modern Kanban-style task board with drag-and-drop functionality.

The application is built using a monorepo architecture with Nx, featuring a NestJS backend and Angular frontend. It demonstrates best practices in software engineering, security, and user experience design.

\vspace{1cm}

\begin{tcolorbox}[colback=green!5!white,colframe=green!75!black,title=Key Features]
\begin{itemize}[leftmargin=*]
    \item Secure JWT-based authentication
    \item Three-tier role-based access control (OWNER, ADMIN, VIEWER)
    \item Complete CRUD operations for task management
    \item Drag-and-drop Kanban board interface
    \item Organization hierarchy for multi-tenant support
    \item Comprehensive audit logging
    \item Responsive design with dark mode support
\end{itemize}
\end{tcolorbox}

\newpage

% Introduction
\section{Introduction}

\subsection{Project Overview}

The Secure Task Management System is a modern web application designed to facilitate team collaboration through efficient task organization and management. The system provides a secure, scalable platform for organizations to manage their workflows with fine-grained access control and comprehensive audit trails.

\subsection{Objectives}

The primary objectives of this project are:

\begin{enumerate}
    \item Implement a secure authentication system using industry-standard JWT tokens
    \item Provide role-based access control with three distinct permission levels
    \item Create an intuitive user interface with drag-and-drop functionality
    \item Ensure data integrity and security through proper validation and authorization
    \item Maintain comprehensive audit logs for compliance and security monitoring
    \item Deliver a responsive, accessible interface that works across all devices
\end{enumerate}

\subsection{Scope}

This project encompasses:

\begin{itemize}
    \item Backend API development with NestJS
    \item Frontend application development with Angular
    \item Database design and implementation with SQLite
    \item Authentication and authorization mechanisms
    \item User interface design and implementation
    \item Testing and quality assurance
    \item Documentation and deployment guidelines
\end{itemize}

\newpage

% Technology Stack
\section{Technology Stack}

\subsection{Backend Technologies}

\begin{table}[H]
\centering
\begin{tabular}{@{}ll@{}}
\toprule
\textbf{Technology} & \textbf{Purpose} \\ \midrule
NestJS & Backend framework \\
TypeORM & Object-Relational Mapping \\
SQLite & Database \\
Passport.js & Authentication middleware \\
JWT & Token-based authentication \\
bcrypt & Password hashing \\
class-validator & Input validation \\
\bottomrule
\end{tabular}
\caption{Backend Technology Stack}
\end{table}

\subsection{Frontend Technologies}

\begin{table}[H]
\centering
\begin{tabular}{@{}ll@{}}
\toprule
\textbf{Technology} & \textbf{Purpose} \\ \midrule
Angular 18+ & Frontend framework \\
TypeScript & Programming language \\
TailwindCSS & Utility-first CSS framework \\
Angular CDK & Component development kit \\
RxJS & Reactive programming \\
\bottomrule
\end{tabular}
\caption{Frontend Technology Stack}
\end{table}

\subsection{Development Tools}

\begin{itemize}
    \item \textbf{Nx:} Monorepo management and build system
    \item \textbf{Jest:} Testing framework
    \item \textbf{ESLint:} Code linting and quality
    \item \textbf{Prettier:} Code formatting
    \item \textbf{Git:} Version control
\end{itemize}

\newpage

% System Architecture
\section{System Architecture}

\subsection{High-Level Architecture}

The application follows a three-tier architecture pattern:

\begin{enumerate}
    \item \textbf{Presentation Layer:} Angular frontend application
    \item \textbf{Application Layer:} NestJS backend API
    \item \textbf{Data Layer:} SQLite database with TypeORM
\end{enumerate}

\subsection{Backend Architecture}

The backend is organized into modular components:

\begin{itemize}
    \item \textbf{AuthModule:} Handles authentication and JWT token generation
    \item \textbf{UsersModule:} Manages user data and operations
    \item \textbf{TasksModule:} Implements task CRUD operations
    \item \textbf{OrganizationsModule:} Manages organizational hierarchy
    \item \textbf{AuditModule:} Tracks and logs user actions
\end{itemize}

\subsection{Frontend Architecture}

The frontend follows Angular best practices with:

\begin{itemize}
    \item \textbf{Core Module:} Services, guards, and interceptors
    \item \textbf{Shared Module:} Reusable components and directives
    \item \textbf{Feature Modules:} Page-specific components
    \item \textbf{Routing Module:} Navigation and route guards
\end{itemize}

\subsection{Communication Flow}

\begin{enumerate}
    \item User interacts with Angular frontend
    \item HTTP requests are sent to NestJS backend
    \item JWT token is attached via HTTP interceptor
    \item Backend validates token and checks permissions
    \item Database operations are performed via TypeORM
    \item Response is sent back to frontend
    \item UI is updated with new data
\end{enumerate}

\newpage

% Database Design
\section{Database Design}

\subsection{Entity Relationship Model}

The database consists of four main entities:

\begin{enumerate}
    \item \textbf{Organization:} Represents a company or team
    \item \textbf{User:} System users with roles and credentials
    \item \textbf{Task:} Work items with status and assignments
    \item \textbf{AuditLog:} Records of user actions
\end{enumerate}

\subsection{Entity Schemas}

\subsubsection{Organization Entity}

\begin{lstlisting}[language=SQL, caption=Organization Table Schema]
CREATE TABLE organization (
    id VARCHAR(36) PRIMARY KEY,
    name VARCHAR(255) NOT NULL,
    createdAt DATETIME DEFAULT CURRENT_TIMESTAMP,
    updatedAt DATETIME DEFAULT CURRENT_TIMESTAMP
);
\end{lstlisting}

\subsubsection{User Entity}

\begin{lstlisting}[language=SQL, caption=User Table Schema]
CREATE TABLE user (
    id VARCHAR(36) PRIMARY KEY,
    username VARCHAR(255) UNIQUE NOT NULL,
    password VARCHAR(255) NOT NULL,
    email VARCHAR(255),
    role ENUM('OWNER', 'ADMIN', 'VIEWER') NOT NULL,
    organizationId VARCHAR(36) NOT NULL,
    createdAt DATETIME DEFAULT CURRENT_TIMESTAMP,
    updatedAt DATETIME DEFAULT CURRENT_TIMESTAMP,
    FOREIGN KEY (organizationId) REFERENCES organization(id)
);
\end{lstlisting}

\subsubsection{Task Entity}

\begin{lstlisting}[language=SQL, caption=Task Table Schema]
CREATE TABLE task (
    id VARCHAR(36) PRIMARY KEY,
    title VARCHAR(255) NOT NULL,
    description TEXT,
    status ENUM('OPEN', 'IN_PROGRESS', 'DONE') NOT NULL,
    organizationId VARCHAR(36) NOT NULL,
    createdById VARCHAR(36) NOT NULL,
    assignedToId VARCHAR(36),
    createdAt DATETIME DEFAULT CURRENT_TIMESTAMP,
    updatedAt DATETIME DEFAULT CURRENT_TIMESTAMP,
    FOREIGN KEY (organizationId) REFERENCES organization(id),
    FOREIGN KEY (createdById) REFERENCES user(id),
    FOREIGN KEY (assignedToId) REFERENCES user(id)
);
\end{lstlisting}

\subsection{Relationships}

\begin{itemize}
    \item One Organization has many Users (1:N)
    \item One Organization has many Tasks (1:N)
    \item One User creates many Tasks (1:N)
    \item One User is assigned to many Tasks (1:N)
    \item Users can have manager-subordinate relationships (M:N)
\end{itemize}

\newpage

% Authentication and Authorization
\section{Authentication and Authorization}

\subsection{Authentication Flow}

\begin{enumerate}
    \item User submits credentials (username and password)
    \item Backend validates credentials against hashed password
    \item If valid, JWT token is generated with user information
    \item Token is returned to frontend
    \item Frontend stores token in localStorage
    \item Token is attached to all subsequent API requests
\end{enumerate}

\subsection{JWT Token Structure}

The JWT token contains:

\begin{lstlisting}[language=JavaScript, caption=JWT Payload]
{
  "sub": "user-uuid",
  "username": "admin",
  "role": "ADMIN",
  "organizationId": "org-uuid",
  "iat": 1234567890,
  "exp": 1234571490
}
\end{lstlisting}

\subsection{Role-Based Access Control}

\subsubsection{Role Hierarchy}

\begin{table}[H]
\centering
\begin{tabular}{@{}lp{10cm}@{}}
\toprule
\textbf{Role} & \textbf{Permissions} \\ \midrule
OWNER & Full system access, user management, all task operations \\
ADMIN & Create and manage tasks, assign tasks to subordinates, view team analytics \\
VIEWER & Read-only access to tasks and analytics \\
\bottomrule
\end{tabular}
\caption{Role Permissions Matrix}
\end{table}

\subsubsection{Permission Enforcement}

Permissions are enforced at multiple levels:

\begin{itemize}
    \item \textbf{Backend Guards:} RolesGuard checks user role before allowing access
    \item \textbf{Frontend Guards:} Route guards prevent unauthorized navigation
    \item \textbf{UI Directives:} *appHasRole directive hides unauthorized UI elements
    \item \textbf{API Validation:} Ownership checks ensure users can only modify their data
\end{itemize}

\newpage

% Features Implementation
\section{Features Implementation}

\subsection{Task Management}

\subsubsection{Create Task}

Users with ADMIN or OWNER roles can create tasks through a modal form:

\begin{lstlisting}[language=TypeScript, caption=Create Task Implementation]
createTask() {
  if (this.createTaskForm.valid) {
    this.taskService.create(this.createTaskForm.value)
      .subscribe(() => {
        this.loadTasks();
        this.closeCreateModal();
      });
  }
}
\end{lstlisting}

\subsubsection{Update Task}

Tasks can be updated via:
\begin{itemize}
    \item Edit modal for title, description, and assignee
    \item Drag-and-drop for status changes
\end{itemize}

\begin{lstlisting}[language=TypeScript, caption=Update Task Implementation]
updateTask() {
  if (this.editTaskForm.valid && this.selectedTask) {
    const updateData = {
      title: this.editTaskForm.value.title,
      description: this.editTaskForm.value.description,
      assigneeId: this.editTaskForm.value.assigneeId
    };
    
    this.taskService.update(this.selectedTask.id, updateData)
      .subscribe(() => {
        this.loadTasks();
        this.closeEditModal();
      });
  }
}
\end{lstlisting}

\subsubsection{Delete Task}

Tasks can be deleted with confirmation:

\begin{lstlisting}[language=TypeScript, caption=Delete Task Implementation]
deleteTask() {
  if (this.selectedTask && 
      confirm('Are you sure you want to delete this task?')) {
    this.taskService.delete(this.selectedTask.id)
      .subscribe(() => {
        this.loadTasks();
        this.closeEditModal();
      });
  }
}
\end{lstlisting}

\subsection{Drag-and-Drop Functionality}

The Kanban board uses Angular CDK for drag-and-drop:

\begin{lstlisting}[language=TypeScript, caption=Drag-and-Drop Handler]
drop(event: CdkDragDrop<Task[]>, status: string) {
  if (event.previousContainer === event.container) {
    moveItemInArray(
      event.container.data, 
      event.previousIndex, 
      event.currentIndex
    );
  } else {
    transferArrayItem(
      event.previousContainer.data,
      event.container.data,
      event.previousIndex,
      event.currentIndex
    );
    const task = event.container.data[event.currentIndex];
    this.taskService.update(task.id, { status })
      .subscribe();
  }
}
\end{lstlisting}

\subsection{Analytics Dashboard}

The analytics component provides:

\begin{itemize}
    \item Total task count
    \item Completion rate calculation
    \item Task distribution by status
    \item Tasks by assignee
    \item Recent activity feed
\end{itemize}

\newpage

% Security Considerations
\section{Security Considerations}

\subsection{Password Security}

\begin{itemize}
    \item Passwords are hashed using bcrypt with salt rounds
    \item Plain text passwords are never stored
    \item Password validation enforces minimum requirements
\end{itemize}

\subsection{Token Security}

\begin{itemize}
    \item JWT tokens have expiration times
    \item Tokens are signed with a secret key
    \item Tokens are validated on every API request
    \item Refresh token mechanism can be implemented
\end{itemize}

\subsection{Input Validation}

\begin{itemize}
    \item All inputs are validated using class-validator
    \item SQL injection prevention through TypeORM parameterization
    \item XSS protection through Angular's built-in sanitization
    \item CORS configuration restricts allowed origins
\end{itemize}

\subsection{Audit Logging}

All user actions are logged with:

\begin{itemize}
    \item User ID and username
    \item Action performed
    \item Resource accessed
    \item Timestamp
    \item Request method and endpoint
\end{itemize}

\newpage

% API Documentation
\section{API Documentation}

\subsection{Authentication Endpoints}

\subsubsection{POST /api/auth/login}

\textbf{Description:} Authenticate user and receive JWT token

\textbf{Request Body:}
\begin{lstlisting}[language=JSON]
{
  "username": "admin",
  "password": "password"
}
\end{lstlisting}

\textbf{Response:}
\begin{lstlisting}[language=JSON]
{
  "access_token": "eyJhbGciOiJIUzI1NiIsInR5cCI6IkpXVCJ9...",
  "user": {
    "id": "uuid",
    "username": "admin",
    "role": "ADMIN"
  }
}
\end{lstlisting}

\subsection{Task Endpoints}

\begin{longtable}{@{}p{2cm}p{4cm}p{7cm}@{}}
\toprule
\textbf{Method} & \textbf{Endpoint} & \textbf{Description} \\ \midrule
\endfirsthead
\toprule
\textbf{Method} & \textbf{Endpoint} & \textbf{Description} \\ \midrule
\endhead
GET & /api/tasks & Get all tasks for user's organization \\
GET & /api/tasks/:id & Get specific task by ID \\
POST & /api/tasks & Create new task (ADMIN/OWNER only) \\
PATCH & /api/tasks/:id & Update task (ADMIN/OWNER only) \\
DELETE & /api/tasks/:id & Delete task (ADMIN/OWNER only) \\
\bottomrule
\caption{Task API Endpoints}
\end{longtable}

\subsection{User Endpoints}

\begin{longtable}{@{}p{2cm}p{4cm}p{7cm}@{}}
\toprule
\textbf{Method} & \textbf{Endpoint} & \textbf{Description} \\ \midrule
\endfirsthead
\toprule
\textbf{Method} & \textbf{Endpoint} & \textbf{Description} \\ \midrule
\endhead
GET & /api/users & Get all users (ADMIN/OWNER only) \\
GET & /api/users/:id & Get specific user by ID \\
POST & /api/users & Create new user (OWNER only) \\
PATCH & /api/users/:id & Update user (OWNER only) \\
DELETE & /api/users/:id & Delete user (OWNER only) \\
\bottomrule
\caption{User API Endpoints}
\end{longtable}

\newpage

% Installation and Setup
\section{Installation and Setup}

\subsection{Prerequisites}

\begin{itemize}
    \item Node.js v18 or higher
    \item npm or pnpm package manager
    \item Git version control
\end{itemize}

\subsection{Installation Steps}

\begin{enumerate}
    \item Clone the repository:
    \begin{lstlisting}[language=bash]
git clone <repository-url>
cd HPuppala-18023069-8027-4aa5-99dc-07eaf624ab5f
    \end{lstlisting}
    
    \item Install dependencies:
    \begin{lstlisting}[language=bash]
npm install
    \end{lstlisting}
    
    \item Start the backend:
    \begin{lstlisting}[language=bash]
npx nx serve api
    \end{lstlisting}
    
    \item Start the frontend (in a new terminal):
    \begin{lstlisting}[language=bash]
npx nx serve dashboard
    \end{lstlisting}
    
    \item Access the application at \url{http://localhost:4200}
\end{enumerate}

\subsection{Default Credentials}

\begin{table}[H]
\centering
\begin{tabular}{@{}lll@{}}
\toprule
\textbf{Username} & \textbf{Password} & \textbf{Role} \\ \midrule
owner & password & OWNER \\
admin & password & ADMIN \\
demo & password & VIEWER \\
\bottomrule
\end{tabular}
\caption{Default User Accounts}
\end{table}

\newpage

% Testing
\section{Testing}

\subsection{Unit Testing}

Run backend unit tests:
\begin{lstlisting}[language=bash]
npx nx test api
\end{lstlisting}

Run frontend unit tests:
\begin{lstlisting}[language=bash]
npx nx test dashboard
\end{lstlisting}

\subsection{Test Coverage}

Generate coverage reports:
\begin{lstlisting}[language=bash]
npx nx test api --coverage
npx nx test dashboard --coverage
\end{lstlisting}

\subsection{End-to-End Testing}

E2E tests can be added using Cypress or Playwright for comprehensive testing of user workflows.

\newpage

% Conclusion
\section{Conclusion}

\subsection{Project Summary}

The Secure Task Management System successfully demonstrates the implementation of a modern, full-stack web application with enterprise-grade features. The project showcases:

\begin{itemize}
    \item Secure authentication and authorization mechanisms
    \item Clean, maintainable code architecture
    \item Modern UI/UX design principles
    \item Comprehensive security measures
    \item Scalable database design
    \item Professional documentation
\end{itemize}

\subsection{Future Enhancements}

Potential improvements for future iterations:

\begin{enumerate}
    \item Real-time updates using WebSockets
    \item Email notifications for task assignments
    \item File attachments for tasks
    \item Advanced search and filtering
    \item Export functionality (PDF, CSV)
    \item Mobile native applications
    \item Integration with third-party services
    \item Advanced analytics and reporting
\end{enumerate}

\subsection{Lessons Learned}

Key takeaways from this project:

\begin{itemize}
    \item Importance of proper security implementation
    \item Value of modular architecture
    \item Benefits of TypeScript for type safety
    \item Effectiveness of monorepo structure
    \item Significance of user experience design
\end{itemize}

\newpage

% References
\section*{References}
\addcontentsline{toc}{section}{References}

\begin{enumerate}
    \item NestJS Documentation. \url{https://docs.nestjs.com/}
    \item Angular Documentation. \url{https://angular.io/docs}
    \item TypeORM Documentation. \url{https://typeorm.io/}
    \item JWT Introduction. \url{https://jwt.io/introduction}
    \item TailwindCSS Documentation. \url{https://tailwindcss.com/docs}
    \item Nx Documentation. \url{https://nx.dev/}
    \item Angular CDK. \url{https://material.angular.io/cdk/categories}
    \item OWASP Security Guidelines. \url{https://owasp.org/}
\end{enumerate}

\newpage

% Appendices
\appendix

\section{Project Structure}

\begin{lstlisting}[basicstyle=\ttfamily\small]
.
├── apps/
│   ├── api/                    # NestJS Backend
│   │   ├── src/
│   │   │   ├── auth/          # Authentication module
│   │   │   ├── users/         # User management
│   │   │   ├── tasks/         # Task management
│   │   │   ├── organizations/ # Organization module
│   │   │   └── audit/         # Audit logging
│   │   └── test/              # Tests
│   └── dashboard/              # Angular Frontend
│       ├── src/
│       │   ├── app/
│       │   │   ├── core/      # Services, guards
│       │   │   ├── pages/     # Components
│       │   │   └── shared/    # Shared code
│       │   └── assets/        # Static files
│       └── tailwind.config.js
├── libs/
│   ├── data/                   # Shared interfaces
│   └── auth/                   # Auth utilities
├── docs/                       # Documentation
├── db.sqlite                   # Database
└── README.md
\end{lstlisting}

\section{Environment Variables}

\begin{lstlisting}[language=bash, caption=.env Configuration]
# Database
DATABASE_PATH=./db.sqlite

# JWT
JWT_SECRET=your-secret-key-here
JWT_EXPIRATION=1d

# Server
PORT=3000
NODE_ENV=development

# Frontend
API_URL=http://localhost:3000
\end{lstlisting}

\section{Glossary}

\begin{description}
    \item[API] Application Programming Interface
    \item[CDK] Component Development Kit
    \item[CRUD] Create, Read, Update, Delete
    \item[JWT] JSON Web Token
    \item[ORM] Object-Relational Mapping
    \item[RBAC] Role-Based Access Control
    \item[REST] Representational State Transfer
    \item[SPA] Single Page Application
    \item[UUID] Universally Unique Identifier
\end{description}

\end{document}
